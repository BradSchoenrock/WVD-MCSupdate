
\section{Nomenclature}

%\begin{itemize}
%\item{Back Panel:\\}
%The portion of the LabVIEW program which contains the wiring diagram and code. 
%\item{Cluster:\\}
%A set of variables saved together in a data structure.
%\item{Control:\\}
%A LabVIEW object which accepts user input and is set before or at runtime. The program can also set this value through a local variable or property node.
%\item{Front Panel:\\}
%The portion of the LabVIEW program which contains user interface and data display. 
%\item{Idle:\\}
%Using a state machine structure, the idle condition is what occurs in the absence of user input after a finite waiting time. In this case, the idle structure performs most typical lidar operations including checking on the data acquisition and reading data.
%\item{Indicator:\\}
%A LabVIEW object which can not be set by the sure at runtime. If the value is set by the user before runtime, the value is overwritten by the program. The program can set this value through a direct call, a local variable, or property node.
%\item{Type Definition:\\}
%A predefined ring loop which can not be redefined without changing the definition itself. This is a way of fixing commonly used terms (like loop stages) such that they can be controlled all at once. 
%\end{itemize}
\newpage    

